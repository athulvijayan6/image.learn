% @Author: Athul Vijayan
% @Date:   2014-05-09 13:56:20
% @Last Modified by:   Athul
% @Last Modified time: 2015-09-29 16:48:39

\documentclass[11pt]{article}
\usepackage[utf8]{inputenc}
\usepackage[a4paper, margin=1in]{geometry}
\usepackage{amsmath}
\usepackage[table]{xcolor}
\usepackage{amssymb}
\usepackage{graphicx}
\usepackage[toc,page]{appendix}
\usepackage{color}
\usepackage{subcaption}
\usepackage[section]{placeins}
\usepackage{rotating}
\usepackage{wrapfig}
\usepackage{bm}
\usepackage[normalem]{ulem}
\usepackage[table]{xcolor}
\newcommand{\HRule}{\rule{\linewidth}{0.2mm}} % Defines a new command for the horizontal lines, change thickness here
\usepackage{hyperref}
\hypersetup{
    colorlinks,
    citecolor=black,
    filecolor=black,
    linkcolor=black,
    urlcolor=cyan
}
\usepackage{array}
\renewcommand{\P}{\mathbb{P}}
\newcolumntype{L}[1]{>{\raggedright\let\newline\\\arraybackslash\hspace{0pt}}m{#1}}
\newcolumntype{C}[1]{>{\centering\let\newline\\\arraybackslash\hspace{0pt}}m{#1}}
\newcolumntype{R}[1]{>{\raggedleft\let\newline\\\arraybackslash\hspace{0pt}}m{#1}}
\newcommand{\rulesep}{\unskip\ \vrule\ }

%----------------------------------------------------------------------------------------
%   TITLE PAGE
%----------------------------------------------------------------------------------------

\newcommand*{\titleGM}{\begingroup % Create the command for including the title page in the document
\hbox{ % Horizontal box
\hspace*{0.2\textwidth} % Whitespace to the left of the title page
\rule{1pt}{\textheight} % Vertical line
\hspace*{0.05\textwidth} % Whitespace between the vertical line and title page text
\parbox[b]{0.75\textwidth}{ % Paragraph box which restricts text to less than the width of the page

{\noindent\Huge\bfseries  Neural data analysis}\\[2\baselineskip] % Title
{\large \textit{Notes}}\\[4\baselineskip] % Tagline or further description
{\Large \textsc{Athul Vijayan \hspace{5pt} ed11b004}} % Author name

\vspace{0.5\textheight} % Whitespace between the title block and the publisher
}}
\endgroup}


\begin{document}
% \titleGM % This command includes the title page
\tableofcontents

% =========================== content =========================
\newpage
\section{Convolutional Neural Networks} % (fold)
CNNs exploit spatially-local correlation by enforcing a local connectivity pattern between neurons of adjacent layers. In other words, the inputs of hidden units in layer $m$ are from a subset of units in layer $m-1$, units that have spatially contiguous receptive fields.

In addition, in CNNs, each filter $h_i$ is replicated across the entire visual field. These replicated units share the same parameterization (weight vector and bias) and form a feature map.

Convolutional Neural Networks take advantage of the fact that the input consists of images and they constrain the architecture in a more sensible way. In particular, unlike a regular Neural Network, the layers of a ConvNet have neurons arranged in 3 dimensions: width, height, depth. (Note that the word depth here refers to the third dimension of an activation volume, not to the depth of a full Neural Network, which can refer to the total number of layers in a network.) For example, the input images in CIFAR-10 are an input volume of activations, and the volume has dimensions 32x32x3 (width, height, depth respectively).

As we described above, a simple ConvNet is a sequence of layers, and every layer of a ConvNet transforms one volume of activations to another through a differentiable function. We use three main types of layers to build ConvNet architectures: Convolutional Layer, Pooling Layer, and Fully-Connected Layer (exactly as seen in regular Neural Networks). We will stack these layers to form a full ConvNet architecture.

Example Architecture: Overview. We will go into more details below, but a simple ConvNet for CIFAR-10 classification could have the architecture [INPUT - CONV - RELU - POOL - FC]. In more detail:

\begin{itemize}
    \item INPUT [32x32x3] will hold the raw pixel values of the image, in this case an image of width 32, height 32, and with three color channels R,G,B.
    \item CONV layer will compute the output of neurons that are connected to local regions in the input, each computing a dot product between their weights and a small region they are connected to in the input volume. This may result in volume such as [32x32x12] if we decided to use 12 filters.
    \item RELU layer will apply an elementwise activation function, such as the max(0,x)max(0,x) thresholding at zero. This leaves the size of the volume unchanged ([32x32x12]).
    \item POOL layer will perform a downsampling operation along the spatial dimensions (width, height), resulting in volume such as [16x16x12].
    \item FC (i.e. fully-connected) layer will compute the class scores, resulting in volume of size [1x1x10], where each of the 10 numbers correspond to a class score, such as among the 10 categories of CIFAR-10. As with ordinary Neural Networks and as the name implies, each neuron in this layer will be connected to all the numbers in the previous volume.
\end{itemize}

 The CONV layer’s parameters consist of a set of learnable filters. Every filter is small spatially (along width and height), but extends through the full depth of the input volume. For example, a typical filter on a first layer of a ConvNet might have size 5x5x3 (i.e. 5 pixels width and height, and 3 because images have depth 3, the color channels). During the forward pass, we slide (more precisely, convolve) each filter across the width and height of the input volume and compute dot products between the entries of the filter and the input at any position. As we slide the filter over the width and height of the input volume we will produce a 2-dimensional activation map that gives the responses of that filter at every spatial position. Intuitively, the network will learn filters that activate when they see some type of visual feature such as an edge of some orientation or a blotch of some color on the first layer, or eventually entire honeycomb or wheel-like patterns on higher layers of the network.

 When dealing with high-dimensional inputs such as images, as we saw above it is impractical to connect neurons to all neurons in the previous volume. Instead, we will connect each neuron to only a local region of the input volume. The spatial extent of this connectivity is a hyperparameter called the receptive field of the neuron (equivalently this is the filter size). The extent of the connectivity along the depth axis is always equal to the depth of the input volume.

% subsection experiment (end)

% ======================= References ==========================
\newpage
\bibliographystyle{plain}
% argument is your BibTeX string definitions and bibliography database(s)
% \bibliography{test}
\bibliography{./bibs}
\end{document}
